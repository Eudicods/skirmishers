Advanced movement includes both additional terrain features, such as water and barbed wire, and the option to shoot while moving.

\paragraph*{Advanced Terrain}

Some maps may offer advanced terrain features that troopers can attempt to navigate.

\begin{table}[H]
\ifthenelse{\not \equal{\outworldsMode}{mode-web}}{\fontfamily{Montserrat-LF}}{\small}\selectfont
\centering
\newcolumntype{R}[1]{>{\raggedleft\let\newline\\\arraybackslash\hspace{0pt}}m{#1}}
\begin{tabular}{!{\Vline{1pt}} m{9em} !{\Vline{1pt}} R{4.5em} !{\Vline{1pt}}}
\Hline{1pt}
\rowcolor{black!30}  \bfseries{Movement Type} & \bfseries{AP Cost} \\
\Hline{1pt}
Change Level      & 3  \\
Rough             & 3  \\
Swamp             & 4  \\
Barbed Wire       & 6  \\
Climb Wall        & 5  \\
Enter/Exit Tunnel & 2  \\
Enter/Exit Trench & 3  \\
Shallow Water     & 4  \\
Swimming          & 4  \\
\Hline{1pt}
\end{tabular}
\caption*{Movement AP Costs}
\end{table}

The map may be marked with terrain height, such as with contour lines indicating elevation.
It costs 3 AP to change elevation by crossing a contour line.

Rough terrain and swamps slow trooper movement but otherwise function the same way as standard movement.
It costs 3 AP to move into a dot in rough terrain and 4 AP to move into a dot in a swamp.
Also, swamp counts as an obstruction for the purposes of resolving attacks.

Barbed wire slows trooper movement and can injure or entangle the trooper.
It costs 6 AP to move onto a dot with barbed wire.
When a trooper moves onto a dot that has barbed wire, apply 1D6 of bludgeoning damage to the trooper as if they received a melee attack.
Then roll 1D6; on a result of 4-6, the trooper is entangled and cannot move further this turn.
On the entangled trooper's next turn, apply 1D6 of additional bludgeoning damage and then roll 1D6 again to determine if the trooper remains entangled.

It costs 2 AP to enter or exit a tunnel and 3 AP to enter or exit a trench.
Moving through a trench costs standard movement AP; however, a trooper in a tunnel must crawl and movement costs double the standard AP.
A trooper in a trench counts as prone when being targeted, but a trooper in a tunnel cannot be seen or targeted by units outside of the tunnel.

Wading through shallow water costs 4 AP but is otherwise like standard movement.
Troopers must swim trough deep water.
A trooper may swim through water if they are only carrying one weapon.
For example, a trooper can only attempt to swim if they are carrying a primary or secondary weapon, but not both.
Note that a grenade launcher attached to a rifle or SMG counts as part of the primary weapon and a trooper may swim with a grenade launcher attached to a rifle or SMG.

Swimming troopers may not fire weapons or use explosives.
It costs 4 AP to move into a water dot.
Explosives do an extra 2 lethal damage (X) at each range to targets in water.

\begin{table}[H]
\ifthenelse{\not \equal{\outworldsMode}{mode-web}}{\fontfamily{Montserrat-LF}}{\small}\selectfont
\centering
\newcolumntype{R}[1]{>{\raggedleft\let\newline\\\arraybackslash\hspace{0pt}}m{#1}}
\begin{tabular}{!{\Vline{1pt}} m{11em} !{\Vline{1pt}} R{4.5em} !{\Vline{1pt}}}
\Hline{1pt}
\rowcolor{black!30}  \bfseries{Condition} & \bfseries{Modifier} \\
\Hline{1pt}
Attacker Entangled & +1  \\
Tunnel Combat      & +2  \\
Defender Swimming  & +1  \\
\Hline{1pt}
\end{tabular}
\caption*{Terrain To Hit Modifiers}
\end{table}

\paragraph*{Movement Fire}

Troopers may use their current active weapon or explosive while moving.
Movement fire attacks are resolved after any incoming attacks as the trooper moves through firing arcs.
Compute the target number as normal, but add a +2 modifier to the target number for movement fire.
After resolving the movement fire attack, the trooper may continue to expend AP for movement.

It costs 2 AP to fire a weapon using movement fire rules.
Single fire weapons can only fire once during movement fire and the trooper cannot establish a firing arc after using movement fire with a single fire weapon.
Burst fire weapons can be fired multiple times with movement fire and the trooper may still establish a firing arc after their movement.

Troopers may use explosives with movement fire.
The explosive must be the current active weapon for the trooper.
The trooper pays the standard AP cost for using the explosive, and the attack is still resolved at the end of the attacking side's turn.
Compute the target number as normal, but add a +2 modifier to the target number for movement fire.

\begin{table}[H]
\ifthenelse{\not \equal{\outworldsMode}{mode-web}}{\fontfamily{Montserrat-LF}}{\small}\selectfont
\centering
\newcolumntype{R}[1]{>{\raggedleft\let\newline\\\arraybackslash\hspace{0pt}}m{#1}}
\begin{tabular}{!{\Vline{1pt}} m{11em} !{\Vline{1pt}} R{4.5em} !{\Vline{1pt}}}
\Hline{1pt}
\rowcolor{black!30}  \bfseries{Condition} & \bfseries{Modifier} \\
\Hline{1pt}
Movement Fire & +2  \\
\Hline{1pt}
\end{tabular}
\caption*{Movement Fire To Hit Modifiers}
\end{table}
