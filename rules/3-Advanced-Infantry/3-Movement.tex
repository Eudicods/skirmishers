Some maps may offer advanced terrain features that troopers can attempt to navigate.

\begin{table}[!h]
\ifthenelse{\not \equal{\outworldsMode}{mode-web}}{\fontfamily{Montserrat-LF}}{\small}\selectfont
\centering
\newcolumntype{R}[1]{>{\raggedleft\let\newline\\\arraybackslash\hspace{0pt}}m{#1}}
\begin{tabular}{!{\Vline{1pt}} m{10em} !{\Vline{1pt}} R{4.5em} !{\Vline{1pt}}}
\Hline{1pt}
\rowcolor{black!30}  \bfseries{Movement Type} & \bfseries{AP Cost} \\
\Hline{1pt}
Change Level      & 3  \\
Rough             & 3  \\
Swamp             & 4  \\
Barbed Wire       & 6  \\
Climb Wall        & 5  \\
Enter/Exit Tunnel & 2  \\
Enter/Exit Trench & 3  \\
Shallow Water     & 4  \\
Swimming          & 4  \\
\Hline{1pt}
\end{tabular}
\caption*{Movement AP Costs}
\end{table}

The map may be marked with terrain height, such as with contour lines indicating elevation.
It costs 3 AP to change elevation by crossing a contour line.

Rough terrain and swamps slow trooper movement but otherwise function the same way as standard movement.
It costs 3 AP to move into a rough dot and 4 AP to move into a swamp dot.
Also, swamp counts as an obstruction for the purposes of resolving attacks.

Barbed wire slows trooper movement and may injure or entangle the trooper.
It costs 6 AP to enter a dot with barbed wire.
When a trooper enters a dot that has barbed wire, apply 1D6 of bludgeoning damage.
Then roll 1D6; on a result of 4-6, the trooper is entangled and cannot move further this turn.
On the trooper's next turn, roll 1D6 again to apply bludgeoning damage, then roll 1D6 to determine if the trooper remains entangled.

It costs 2 AP to enter or exit a tunnel and 3 AP to enter or exit a trench.
Moving through a trench costs standard movement AP; however, a trooper in a tunnel must crawl and movement costs double the standard AP.
A trooper in a trench counts as prone when being targeted, but a trooper in a tunnel cannot be seen or targeted by units outside of the tunnel.

Wading through shallow water costs 4 AP but is otherwise like standard movement.
Troopers must swim trough deep water.
A trooper may swim across water if they are only carrying one weapon.
For example, a trooper can only attempt to swim if they are carrying a primary or secondary weapon, but not both.
Note that a grenade launcher attached to a rifle or SMG counts as part of the primary weapon and a trooper may swim with a grenade launcher attached to a rifle or SMG.

Swimming troopers may not fire weapons or use explosives.
It costs 4 AP to move into a water dot.
Explosives do an extra 2 standard damage at each range to targets in water.

\begin{table}[!h]
\ifthenelse{\not \equal{\outworldsMode}{mode-web}}{\fontfamily{Montserrat-LF}}{\small}\selectfont
\centering
\newcolumntype{R}[1]{>{\raggedleft\let\newline\\\arraybackslash\hspace{0pt}}m{#1}}
\begin{tabular}{!{\Vline{1pt}} m{11em} !{\Vline{1pt}} R{4.5em} !{\Vline{1pt}}}
\Hline{1pt}
\rowcolor{black!30}  \bfseries{Condition} & \bfseries{Modifier} \\
\Hline{1pt}
Attacker Entangled & +1  \\
Tunnel Combat      & +2  \\
Defender Swimming  & +1  \\
\Hline{1pt}
\end{tabular}
\caption*{To Hit Modifiers}
\end{table}

\paragraph*{Jet Packs}

Troopers equipped with jet packs may move further than standard troopers.
Jet packs are heavy, and troopers wearing jet packs receive a -2 AP penalty.

It costs 3 AP to activate a jet pack.
When activating a jet pack, a trooper must move 6 to 16 dots.
A jet pack cannot be used to move fewer than 6 dots, and a jet pack can only be used once in a turn.
A jet pack can only be used outdoors or next to a window or door if the trooper immediately goes outside during the flight.

A trooper can go up or down in elevation 1 level for every 3 dots they travel.
For example, if a trooper activated a jet pack to move 10 dots, then they can go up or down in elevation by up to 3 levels.
A trooper on the ground could land on the roof of a two story building by moving at least 6 dots when activating their jet pack.

\begin{table}[!h]
\ifthenelse{\not \equal{\outworldsMode}{mode-web}}{\fontfamily{Montserrat-LF}}{\small}\selectfont
\centering
\newcolumntype{R}[1]{>{\raggedleft\let\newline\\\arraybackslash\hspace{0pt}}m{#1}}
\begin{tabular}{!{\Vline{1pt}} m{10em} !{\Vline{1pt}} R{4.5em} !{\Vline{1pt}}}
\Hline{1pt}
\rowcolor{black!30}  \bfseries{Condition} & \bfseries{Modifier} \\
\Hline{1pt}
Defender Flying & +2  \\
\Hline{1pt}
\end{tabular}
\caption*{To Hit Modifiers}
\end{table}

If a trooper flies through an enemy firing arc, the trooper may be fired upon.
Apply a +2 modifier when targeting a flying trooper.
If the trooper receives damage that reduces their AP, they immediately fall prone on the ground.

A flying trooper may drop a hand grenade, if they have one ready.
The hand grenade may be dropped on any dot the trooper flies over.
Use 8 as the base target number and resolve the attack as usual.

A trooper may remove or equip a jet pack for 5 AP, and a jet pack may be exchanged between troopers for 6 AP, like with weapons or ammunition.
