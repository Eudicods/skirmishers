Troopers carry one primary weapon, one secondary weapon, and 4 hand grenades.
All troopers have a helmet and combat knife.
The combat knife is assumed to be affixed as a bayonet if the trooper is using some type of rifle or SMG.
In some cases, troopers may carry additional equipment.

\paragraph*{Primary Weapons}

Troopers may use the following primary weapons:

\begin{table}[!h]
\ifthenelse{\not \equal{\outworldsMode}{mode-web}}{\fontfamily{Montserrat-LF}}{\small}\selectfont
\centering
\newcolumntype{R}[1]{>{\raggedleft\let\newline\\\arraybackslash\hspace{0pt}}m{#1}}
\begin{tabular}{!{\Vline{1pt}} m{6em} !{\Vline{1pt}} R{4.5em} !{\Vline{1pt}} R{7em} !{\Vline{1pt}} R{7em} !{\Vline{1pt}}}
\Hline{1pt}
\rowcolor{black!30}  \bfseries{Weapon} & \bfseries{Damage} & \bfseries{Short Range} & \bfseries{Long Range} \\
\Hline{1pt}
Auto Pistol*  & 2X & 1-7  & 8-20   \\
Rifle         & 4X & 1-27 & 28-75  \\
SMG*          & 3X & 1-17 & 18-25  \\
Laser Rifle   & 5X & 1-17 & 18-105 \\
Laser SMG*    & 4X & 1-25 & 26-70  \\
Gyrojet Rifle & 6X & 1-57 & 58-180 \\
\Hline{1pt}
\end{tabular}
\caption*{Primary Weapons}
\end{table}

Note, the auto pistol, SMG, and Laser SMG can fire into the same firing arc multiple times.
All other weapons may only fire into a firing arc once per turn.

\paragraph*{Secondary Weapons}

Each trooper can have a maximum of 1 secondary weapon.
Troopers may use the following ranged secondary weapons:

\begin{table}[!h]
\ifthenelse{\not \equal{\outworldsMode}{mode-web}}{\fontfamily{Montserrat-LF}}{\small}\selectfont
\centering
\newcolumntype{R}[1]{>{\raggedleft\let\newline\\\arraybackslash\hspace{0pt}}m{#1}}
\begin{tabular}{!{\Vline{1pt}} m{6em} !{\Vline{1pt}} R{4.5em} !{\Vline{1pt}} R{7em} !{\Vline{1pt}} R{7em} !{\Vline{1pt}}}
\Hline{1pt}
\rowcolor{black!30}  \bfseries{Weapon} & \bfseries{Damage} & \bfseries{Short Range} & \bfseries{Long Range} \\
\Hline{1pt}
Pistol        & 3X & 1-7  & 8-20  \\
Laser Pistol  & 4X & 1-12 & 13-30 \\
\Hline{1pt}
\end{tabular}
\caption*{Ranged Secondary Weapons}
\end{table}

Instead of a ranged secondary weapon, troopers may use the following hand to hand secondary weapons:

\begin{table}[!h]
\ifthenelse{\not \equal{\outworldsMode}{mode-web}}{\fontfamily{Montserrat-LF}}{\small}\selectfont
\centering
\newcolumntype{R}[1]{>{\raggedleft\let\newline\\\arraybackslash\hspace{0pt}}m{#1}}
\begin{tabular}{!{\Vline{1pt}} m{7.2em} !{\Vline{1pt}} R{5em} !{\Vline{1pt}} R{7em} !{\Vline{1pt}} R{7em} !{\Vline{1pt}}}
\Hline{1pt}
\rowcolor{black!30}  \bfseries{Weapon} & \bfseries{Damage} & \bfseries{Short Range} & \bfseries{Long Range} \\
\Hline{1pt}
Fists         & (AP/2)\textbackslash & 1 & - \\
Blackjack     & 5\textbackslash      & 1 & - \\
Club          & 1X, 4\textbackslash  & 1 & - \\
Stun Baton    & 8\textbackslash      & 1 & - \\
Bayonet/Knife & 3X                   & 1 & - \\
Sword         & 4X                   & 1 & - \\
Vibroblade    & 5X                   & 1 & - \\
\Hline{1pt}
\end{tabular}
\caption*{Hand to Hand Secondary Weapons}
\end{table}

Troopers may use the following explosives:

\begin{table}[!h]
\ifthenelse{\not \equal{\outworldsMode}{mode-web}}{\fontfamily{Montserrat-LF}}{\small}\selectfont
\centering
\newcolumntype{R}[1]{>{\raggedleft\let\newline\\\arraybackslash\hspace{0pt}}m{#1}}
\begin{tabular}{!{\Vline{1pt}} m{12.1em} !{\Vline{1pt}} R{4.9em} !{\Vline{1pt}} R{4.5em} !{\Vline{1pt}} R{4.5em} !{\Vline{1pt}} R{7em} !{\Vline{1pt}} R{7em} !{\Vline{1pt}}}
\Hline{1pt}
\rowcolor{black!30}  \bfseries{Weapon} & \bfseries{Damage} & \bfseries{AP} & \bfseries{Ammo} & \bfseries{Short Range} & \bfseries{Long Range} \\
\Hline{1pt}
Satchel Charge          &   10X/5X/2X & 3 or AP & 1 & AP/2 & -      \\
Hand Grenade            &    6X/3X/1X & AP      & 4 & AP   & -      \\
\Hline{1pt}
\end{tabular}
\caption*{Explosive Weapons}
\end{table}

The ammo indicates how many rounds come with the weapon.
The damage is given is descending order for the point of detonation, the adjacent dots, the dots 2 away from the detonation, and so on.
It takes 1 AP to throw a hand grenade 1 dot, and it takes 2 AP to throw a satchel charge 1 dot.

\paragraph*{Armor}

Troopers may equip body armor.
Body armor protects a number of contiguous cells in the armor section of the unit card.
Cells protected by body armor ignore bludgeoning damage.
Standard damage is marked like bludgeoning damage on the cell of body armor.
The first hit with standard damage partially marks the cell (\textbackslash) and the second hit fully marks the cell (X).

\begin{table}[!h]
\ifthenelse{\not \equal{\outworldsMode}{mode-web}}{\fontfamily{Montserrat-LF}}{\small}\selectfont
\centering
\newcolumntype{R}[1]{>{\raggedleft\let\newline\\\arraybackslash\hspace{0pt}}m{#1}}
\begin{tabular}{!{\Vline{1pt}} m{10em} !{\Vline{1pt}} R{4.5em} !{\Vline{1pt}} R{4.5em} !{\Vline{1pt}}}
\Hline{1pt}
\rowcolor{black!30}  \bfseries{Armor Type} & \bfseries{Coverage} & \bfseries{AP Cost} \\
\Hline{1pt}
Light & 2 & 1  \\
Basic & 4 & 2  \\
Heavy & 6 & 3  \\
\Hline{1pt}
\end{tabular}
\caption*{Body Armor AP Costs}
\end{table}

To equip body armor on a trooper, mark the cells covered by body armor on the unit card.
Then, reduce the AP to account for the body armor.
For example, with basic body armor, mark 4 contiguous cells as protected by armor and reduce all values in the AP row by 2.
