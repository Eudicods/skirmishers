After moving, troopers may use any remaining AP for combat actions, such as establishing a firing arc, engaging in hand to hand combat, or using explosives.

\paragraph*{Firing Arcs}

A trooper may establish a 30 degree, 60 degree, or 150 degree firing arc by paying the corresponding AP.
When establishing a firing arc, the vertex of the arc is the dot the trooper is standing on and can be oriented in any direction.
The trooper may make a fixed number of shots into the firing arc, given by the base number of shots for the firing arc multiplied by the weapon modifier.
For example, a 60 degree firing arc has a base of 2 shots.
A trooper with a pistol may shoot into this arc up to 4 times, while a trooper with a SMG may shoot into this arc 6 times.

When an enemy trooper activates in or moves into the firing arc, the firing trooper may fire upon the targeted trooper.
This attack is resolved before the targeted trooper continues their movement.

\begin{table}[H]
\ifthenelse{\not \equal{\outworldsMode}{mode-web}}{\fontfamily{Montserrat-LF}}{\small}\selectfont
\centering
\newcolumntype{R}[1]{>{\raggedleft\let\newline\\\arraybackslash\hspace{0pt}}m{#1}}
\begin{tabular}{!{\Vline{1pt}} m{8em} !{\Vline{1pt}} R{4.5em} !{\Vline{1pt}} R{4.5em} !{\Vline{1pt}}}
\Hline{1pt}
\rowcolor{black!30}  \bfseries{Firing Arc Size} & \bfseries{Shots} & \bfseries{AP Cost} \\
\Hline{1pt}
30 Degrees  & 1 & 2  \\
60 Degrees  & 2 & 4  \\
150 Degrees & 4 & 6  \\
\Hline{1pt}
\end{tabular}
\caption*{Firing Arc AP Costs}
\end{table}

\paragraph*{Line of Sight}

If a trooper can only attack a target if they have a valid line of sight.
Draw a straight line between the firing trooper and the targeted trooper.
Obstructions, such as furniture, light vegetation, and low walls, provide cover but do not block line of sight.
Additional troopers, enemy or friendly, between the firing trooper and the targeted trooper also provide cover, unless the firing trooper is standing and the intervening trooper is prone.
Impassible obstacles, such as a solid high wall, the truck of a tree, or a trooper, block line of sight.

If the firing trooper or targeted trooper is prone, then any solid low obstructions, such as furniture and low walls, that are adjacent to firing trooper block line of sight.
If the firing trooper and the targeted trooper are prone, then any solid low obstructions block line of sight.

If either the firing trooper or the targeted trooper is at a higher level, such as at the window in a building, then only obstructions and obstacles at the same level of the higher trooper provide cover or block line of sight.
However, all obstructions and obstacles within one dot of the targeted trooper always provides cover or blocks line of sight.

\paragraph*{Resolving Fire}

Note: Players resolve fire during the enemy movement, which occurs during your opponent's turn.

If an enemy trooper activates in or moves into a firing arc, the firing trooper may choose to fire as long as the firing trooper has a valid line of sight.
If an enemy trooper is in multiple firing arcs, then firing troopers may all choose to fire, in any order.
If an enemy trooper stands from prone, then they may be shot by any firing trooper that did not previously attack them on the current dot.
A firing trooper may decide to not fire if they wish to preserve the firing arc for a future target.

Resolving fire is a 2D6 check with the firing trooper's current active weapon.
If the roll meets or exceeds the target number, then the attack succeeds.

The base target number for ranged attacks is 6.
Count the number of dots in the shortest path between the firing trooper and target trooper to determine the range.
There is no modifier to the target number if the weapon is in short range.
If the weapon is in long range, then add a +2 modifier to the target number.

Apply any modifier for damage to the firing trooper.
Add a +1 modifier for each obstruction in the path of fire between the firing trooper and the target trooper, except for any obstruction between the firing trooper and any adjacent dots.
Use a -1 modifier if the firing trooper is prone.
Add a +1 modifier if the targeted trooper is prone and greater than 1 dot away.
Use a -1 modifier if the targeted trooper is prone and 1 dot away.

\begin{table}[H]
\ifthenelse{\not \equal{\outworldsMode}{mode-web}}{\fontfamily{Montserrat-LF}}{\small}\selectfont
\centering
\newcolumntype{R}[1]{>{\raggedleft\let\newline\\\arraybackslash\hspace{0pt}}m{#1}}
\begin{tabular}{!{\Vline{1pt}} m{12em} !{\Vline{1pt}} R{4.5em} !{\Vline{1pt}}}
\Hline{1pt}
\rowcolor{black!30}  \bfseries{Condition} & \bfseries{Modifier} \\
\Hline{1pt}
Short Range           & +0      \\
Long Range            & +2      \\
Cover                 & +1 per  \\
Attacker Prone        & --1     \\
Target Prone          & +1      \\
Target Prone Adjacent & --1     \\
\Hline{1pt}
\end{tabular}
\caption*{Target Number Modifiers}
\end{table}

If the attack succeeds, roll an additional 2D6.
This is the initial cell to apply damage in.

Each weapon has a lethal damage value (X) and a bludgeoning damage value (\textbackslash).
Lethal damage is applied first.
If the trooper has any remaining body armor, cross out body armor cell for each point of lethal damage.
If any lethal damage remains after destroying any body armor, then apply the remaining damage to the troopers HP cells.
Start with the initial cell given by the 2D6 roll and fully cross out the number of HP cells given by the lethal damage value (X) of the weapon.
Skip any previously fully crossed out cells and fully cross out any partially crossed out or undamaged HP cells.

Then apply any bludgeoning damage (\textbackslash).
If the trooper has any remaining body armor, then ignore all bludgeoning damage.
Start with the first cell after the last cell affected by the lethal damage or the initial cell if the weapon does not have a lethal damage value (X).
Partially cross out the number of HP cells given by the bludgeoning damage value (\textbackslash) of the weapon.
If an HP cell is already partially crossed out, then fully cross out the cell.

The targeted trooper's new HP is given by the first cell to the right of the lowest fully crossed out HP cell.
The corresponding maximum AP is given by this column on the unit card.
The new maximum AP immediately applies to the targeted trooper.
If this new AP value meets or exceeds the AP the targeted trooper has already spent this turn, the targeted trooper immediately falls prone and can take no further actions.

If the firing trooper has fired the maximum number of shots the weapon supports, then remove the firing arc once the firing is resolved.
A trooper may decide to not fire if they wish to preserve the firing arc for a future target.

\paragraph*{Melee Combat}

It costs 5 AP to make a melee attack.
The amount of bludgeoning damage for a melee attack with fists is given by half the attacking trooper's current maximum AP.
Melee combat is resolved in the same way as ranged combat, and the attack is resolved during the targeted trooper's turn when they activate, before the targeted trooper takes any actions or movement.

The attacking trooper must have a melee weapon active.
The attacking trooper may use a bayonet attached to their current active weapon or use their active weapon as a club.
A trooper can always make a melee attack with their fists.

Resolving melee combat is a 2D6 check with the attacking trooper's current active weapon.
If the roll meets or exceeds the target number, then the attack succeeds.

The base target number for melee attacks is 4.
Apply all appropriate modifiers given above, except any modifiers for cover.
Damage is applied as above.

\paragraph*{Explosives}

AP cost for attacks with explosives are determined by the weapon.
A thrown grenade costs 1 AP for each dot its thrown, and a thrown satchel charge costs 2 AP for each dot its thrown.
Double these AP costs if the attacking trooper is prone.
A trooper cannot set up a firing arc after using an explosive.
The explosive must be the trooper's current active weapon in order to make an attack with an explosive.
A trooper automatically switches back to their primary or secondary weapon after using a grenade or satchel charge, for 0 AP.

A trooper can throw a hand grenade or satchel charge down a ladder or stairs any number of levels or up one level, at a cost of 3 AP.
The trooper must be standing on the stairwell or ladder to throw the explosive up a level and may be on or adjacent to the stairwell or ladder to throw the explosive down any number of levels.

Explosives are resolved in the same way as ranged combat, but the attack is resolved at the end of the attacking side's turn.
Place a token at the targeted dot for the explosives during the firing trooper's turn and resolve the explosive after all troopers on your side have activated.

Resolving explosives is a 2D6 check with the firing trooper's current active explosive.
If the roll meets or exceeds the target number, then the attack succeeds.

The base target number is the number of AP used to throw the hand grenade or satchel charge.
Apply all appropriate modifiers given above.
Since explosives target a dot and not a trooper, do not apply any modifiers for a prone target.

If the target number is met, then apply damage starting at the targeted dot and moving outwards.
Damage is applied as above.
Impassible obstacles between the center of the explosion and the affected trooper, such as a solid high wall or the truck of a tree, block explosive damage.
Low solid obstructions such as furniture and windows also block explosive damage if the affected trooper is prone.
Explosive damage is also applied to troopers above or below the explosion, such as in a stairwell.
Treat each level of height as one dot further away from the center of the explosion.

If the target number is not met, then the explosive scatters.
The explosive scatters from the targeted dot.
If the explosive was fired through a window, then the explosive scatters from the first dot before the window along the line of sight.
Roll 1D6 to determine the scatter direction, identifying one direction to correspond to a result of 1 and proceeding clockwise with the other values.
If the path of scatter intersects an impassible obstacle, such as a solid high wall, then the explosive does not scatter and the center of the explosion stays on the targeted dot.
Move the center of the explosion to the new dot and apply damage as above.
