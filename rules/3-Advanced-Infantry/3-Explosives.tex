Troopers may use the following advanced explosives:

\begin{table}[!h]
\ifthenelse{\not \equal{\outworldsMode}{mode-web}}{\fontfamily{Montserrat-LF}}{\small}\selectfont
\centering
\newcolumntype{R}[1]{>{\raggedleft\let\newline\\\arraybackslash\hspace{0pt}}m{#1}}
\begin{tabular}{!{\Vline{1pt}} m{12.1em} !{\Vline{1pt}} R{4.5em} !{\Vline{1pt}} R{4.5em} !{\Vline{1pt}} R{4.5em} !{\Vline{1pt}} R{7em} !{\Vline{1pt}} R{7em} !{\Vline{1pt}}}
\Hline{1pt}
\rowcolor{black!30}  \bfseries{Weapon} & \bfseries{Damage} & \bfseries{AP} & \bfseries{Ammo} & \bfseries{Short Range} & \bfseries{Long Range} \\
\Hline{1pt}
Light Anti-Tank Weapon  &    7/3/1 & 4 & 1 & 1-22 & 23-80  \\
Rocket Launcher         &    8/4/2 & 5 & 2 & 1-44 & 45-108 \\
Anti-Tank Missile       & 13/6/3/1 & 5 & 1 & 1-54 & 55-96  \\
Grenade Launcher        &    4/2/1 & 3 & 4 & 1-10 & 11-24  \\
Auto-Grenade Launcher   &    3/2/1 & 4 & 4 & 1-22 & 23-50  \\
Recoilless Rifle        &    4/2/1 & 4 & 2 & 1-36 & 37-70  \\
\Hline{1pt}
\end{tabular}
\caption*{Advanced Explosive Weapons}
\end{table}

The ammo indicates how many rounds come with the weapon.
The damage is given is descending order for the point of detonation, the adjacent dots, the dots 2 away from the detonation, and so on.

Rocket launchers, anti-tank missiles, and recoilless rifles are primary weapons.
Light anti-tank weapons are secondary weapons.

A trooper must be using a rifle or SMG as their primary weapon to use a grenade launcher.
The launcher takes the place of a secondary weapon, but it is mounted under the body of the rifle or SMG.
A trooper does not have to pay AP to switch between their primary weapon and a launcher.
An auto-grenade launcher is a separate weapon and cannot be mounted under a primary weapon in this fashion.

In the Other section of the unit card, a trooper may carry extra ammunition, doubling the ammunition available for any explosive weapon.

Advanced explosive weapons resolve in the same way as hand grenades and satchel charges, but these weapons only scatter 1 dot.
