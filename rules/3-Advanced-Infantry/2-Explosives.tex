Troopers may use the following advanced explosives:

\begin{table}[!h]
\ifthenelse{\not \equal{\outworldsMode}{mode-web}}{\fontfamily{Montserrat-LF}}{\small}\selectfont
\centering
\newcolumntype{R}[1]{>{\raggedleft\let\newline\\\arraybackslash\hspace{0pt}}m{#1}}
\begin{tabular}{!{\Vline{1pt}} m{12.1em} !{\Vline{1pt}} R{6.2em} !{\Vline{1pt}} R{4.1em} !{\Vline{1pt}} R{4em} !{\Vline{1pt}} R{6.5em} !{\Vline{1pt}} R{6.5em} !{\Vline{1pt}}}
\Hline{1pt}
\rowcolor{black!30}  \bfseries{Weapon} & \bfseries{Damage} & \bfseries{AP Cost} & \bfseries{Ammo} & \bfseries{Short Range} & \bfseries{Long Range} \\
\Hline{1pt}
Light Anti-Tank Weapon  &     7X/3X/1X & 4 & 1 & 1-22 & 23-80  \\
Rocket Launcher         &     8X/4X/2X & 5 & 2 & 1-44 & 45-108 \\
Anti-Tank Missile       & 13X/6X/3X/1X & 5 & 1 & 1-54 & 55-96  \\
Grenade Launcher        &     4X/2X/1X & 3 & 4 & 1-10 & 11-24  \\
Auto-Grenade Launcher   &     3X/2X/1X & 4 & 4 & 1-22 & 23-50  \\
Recoilless Rifle        &     4X/2X/1X & 4 & 2 & 1-36 & 37-70  \\
\Hline{1pt}
\end{tabular}
\caption*{Advanced Explosive Weapons}
\end{table}

The ammo indicates how many rounds come with the weapon.
The damage is given is descending order for the point of detonation, the adjacent dots, the dots 2 away from the detonation, and so on.

Rocket launchers, anti-tank missiles, grenade launchers, auto-grenade launchers, and recoilless rifles are primary weapons.
Light anti-tank weapons are secondary weapons.

A trooper may carry a grenade launcher as a primary weapon or mount a grenade launcher under a rifle or SMG.
If mounted under the rifle or SMG, the launcher takes the place of a secondary weapon.
A trooper does not have to pay AP to switch between their primary weapon and a launcher.
An auto-grenade launcher is a separate weapon and cannot be mounted under a rifle or SMG in this fashion.

In the Other section of the unit card, a trooper may carry extra ammunition, doubling the ammunition available for any explosive weapon.

Advanced explosive weapons resolve in the same way as hand grenades and satchel charges, but these weapons only scatter 1 dot.

\paragraph*{Incendiary Weapons}

Incendiary weapons are resolved like explosives.

\begin{table}[!h]
\ifthenelse{\not \equal{\outworldsMode}{mode-web}}{\fontfamily{Montserrat-LF}}{\small}\selectfont
\centering
\newcolumntype{R}[1]{>{\raggedleft\let\newline\\\arraybackslash\hspace{0pt}}m{#1}}
\begin{tabular}{!{\Vline{1pt}} m{12.1em} !{\Vline{1pt}} R{5em} !{\Vline{1pt}} R{4.1em} !{\Vline{1pt}} R{4em} !{\Vline{1pt}} R{6.5em} !{\Vline{1pt}} R{6.5em} !{\Vline{1pt}}}
\Hline{1pt}
\rowcolor{black!30}  \bfseries{Weapon} & \bfseries{Damage} & \bfseries{AP Cost} & \bfseries{Ammo} & \bfseries{Short Range} & \bfseries{Long Range} \\
\Hline{1pt}
Flamethrower       &    2X/1X &  2 & - &  1-6 &  7-12  \\
Heavy Flamethrower & 2X/2X/1X &  4 & 4 &  1-6 &  7-12  \\
Incendiary Grenade & 4X/2X/1X & AP & 4 &   AP &   -    \\
Incendiary Rocket  & 4X/2X/1X &  5 & 2 & 1-44 & 45-108 \\
\Hline{1pt}
\end{tabular}
\caption*{Advanced Explosive Weapons}
\end{table}

The ammo indicates how many rounds come with the weapon.
A flamethrower has effectively unlimited ammunition.
The damage is given is descending order for the point of detonation, the adjacent dots, the dots 2 away from the detonation, and so on.

As with explosives, impassible obstacles, such as a solid high wall or the truck of a tree, block incendiary damage.
Solid obstructions such as furniture and windows also block incendiary damage.

All effected dots light on fire after resolving the damage.
Fire is an obstruction that is as tall as a trooper and provides cover, like light vegetation.
Troopers entering a dot on fire take 1 point of standard damage (X).
If a trooper ends their movement on a dot that is on fire, they take 8 points of standard damage (X).

Incendiary grenades are resolved like hand grenades but do incendiary damage, as described above.

\paragraph*{Smoke Grenades}

Troopers may replace any number of their 4 hand grenades with smoke grenades.
Smoke grenades are resolved like hand grenades but create some instead of explosive damage.

After resolving the location the smoke grenade lands, mark the dot where the smoke grenade explodes.
This dot and all adjacent dots are filled with smoke.
Smoke is an obstruction like light vegetation, but it provides a +3 modifier.
At the end of the next movement phase for your side, expand the smoke to cover all dots within 2 of the location where the smoke grenade landed.
Smoke stops if it encounters an impassible obstacle, such as a solid high wall.
Remove the smoke after 12 turns.

\begin{table}[!h]
\ifthenelse{\not \equal{\outworldsMode}{mode-web}}{\fontfamily{Montserrat-LF}}{\small}\selectfont
\centering
\newcolumntype{R}[1]{>{\raggedleft\let\newline\\\arraybackslash\hspace{0pt}}m{#1}}
\begin{tabular}{!{\Vline{1pt}} m{10em} !{\Vline{1pt}} R{4.5em} !{\Vline{1pt}}}
\Hline{1pt}
\rowcolor{black!30}  \bfseries{Condition} & \bfseries{Modifier} \\
\Hline{1pt}
Smoke & +3  \\
\Hline{1pt}
\end{tabular}
\caption*{To Hit Modifiers}
\end{table}
