There are several optional rules for combat.

\paragraph*{Large Targets}

Units may target large objects, such as buildings, walls, or vehicles, with explosives.
Apply a -2 modifier when attacking a large object.

\begin{table}[!h]
\ifthenelse{\not \equal{\outworldsMode}{mode-web}}{\fontfamily{Montserrat-LF}}{\small}\selectfont
\centering
\newcolumntype{R}[1]{>{\raggedleft\let\newline\\\arraybackslash\hspace{0pt}}m{#1}}
\begin{tabular}{!{\Vline{1pt}} m{7em} !{\Vline{1pt}} R{4.5em} !{\Vline{1pt}}}
\Hline{1pt}
\rowcolor{black!30}  \bfseries{Condition} & \bfseries{Modifier} \\
\Hline{1pt}
Large Target & -2  \\
\Hline{1pt}
\end{tabular}
\caption*{To Hit Modifiers}
\end{table}

If the cumulative damage applied to a wall or building exceeds the toughness of the structure, then the structure is breached.
Damaged is tracked per dot.
The location of the breach becomes rough terrain and costs 3 AP to move into.
If using breach rules, the toughness of each building and wall must be identified before the game starts.
The default toughness of a building is medium.

\begin{table}[!h]
\ifthenelse{\not \equal{\outworldsMode}{mode-web}}{\fontfamily{Montserrat-LF}}{\small}\selectfont
\centering
\newcolumntype{R}[1]{>{\raggedleft\let\newline\\\arraybackslash\hspace{0pt}}m{#1}}
\begin{tabular}{!{\Vline{1pt}} m{6em} !{\Vline{1pt}} R{6em} !{\Vline{1pt}}}
\Hline{1pt}
\rowcolor{black!30}  \bfseries{Building} & \bfseries{Toughness} \\
\Hline{1pt}
Tree     &  5   \\
Wall     & 10   \\
Light    & 15   \\
Medium   & 40   \\
Heavy    & 90   \\
Hardened & 140  \\
\Hline{1pt}
\end{tabular}
\caption*{Building Toughness}
\end{table}

Explosives may also be used to target trees.
As with buildings, after the tree is destroyed, the terrain becomes rough at that dot and costs 3 AP to move into.

\paragraph*{Trooper Experience}

The basic rules assume regular troopers; however, troopers may have different experience levels.
More experienced troopers receive extra AP during their turn, but troopers may still only spend a maximum of 8 AP on movement.

\begin{table}[!h]
\ifthenelse{\not \equal{\outworldsMode}{mode-web}}{\fontfamily{Montserrat-LF}}{\small}\selectfont
\centering
\newcolumntype{R}[1]{>{\raggedleft\let\newline\\\arraybackslash\hspace{0pt}}m{#1}}
\begin{tabular}{!{\Vline{1pt}} m{6.5em} !{\Vline{1pt}} R{5.5em} !{\Vline{1pt}}}
\Hline{1pt}
\rowcolor{black!30}  \bfseries{Experience} & \bfseries{AP Bonus} \\
\Hline{1pt}
Green   & -1  \\
Regular &  0  \\
Veteran & +1  \\
Elite   & +2  \\
\Hline{1pt}
\end{tabular}
\caption*{Experience AP Bonus}
\end{table}

\paragraph*{Morale Checks}

Once 4 or more troopers in a squad have died, the remaining troopers must make morale checks.
Each remaining trooper must roll 2D6 and meet the number given in the chart below.
Apply a +1 modifier if more than 4 troopers are dead.
For example, add a +2 modifier if 6 troopers are dead.
Apply a -N modifier based upon the leadership skill of the squad leader, if the squad leader is alive.
The standard leadership skill of a squad leader is 2.

\begin{table}[!h]
\ifthenelse{\not \equal{\outworldsMode}{mode-web}}{\fontfamily{Montserrat-LF}}{\small}\selectfont
\centering
\newcolumntype{R}[1]{>{\raggedleft\let\newline\\\arraybackslash\hspace{0pt}}m{#1}}
\begin{tabular}{!{\Vline{1pt}} m{6.5em} !{\Vline{1pt}} R{4em} !{\Vline{1pt}}}
\Hline{1pt}
\rowcolor{black!30}  \bfseries{Experience} & \bfseries{Target} \\
\Hline{1pt}
Green   & 9  \\
Regular & 7  \\
Veteran & 5  \\
Elite   & 2  \\
\Hline{1pt}
\end{tabular}
\caption*{Morale Checks}
\end{table}
